\section{Wertung}
Im Folgenden wird eine Zusammenfassung sowie eine Einsicht in die Bedeutung der zuvor besprochenen Kapitel gegeben.\\

In der Praxis ist der Einsatz von Signalverarbeitung unumgänglich und ein Schritt von zunehmend größer werdender Bedeutung. Durch das vermehrte Aufkommen von Sensoren in allerlei Form, unter anderem Kameras, Wettererfassung (Luftfeuchtigkeit, Temperatur, Niederschlag, Sonnenstunden, Windstärke, etc.), Mikrofone sowie Implementierung diverse Arten von Sensoren in Mobilen Geräten, steigt die Nachfrage nach kleineren, schnelleren, genaueren, energiesparenderen und günstigeren Formen der Signalerfassung sowie deren Verarbeitung.\\
Unter Betrachtung des Abtasttheorems ist es offensichtlich, dass die realisierbare Abtastfrequenz die Signalerfassung stark limitiert. Durch die Beschränkung aus dem Bereich der Mikroprozessoren sowie den in [\ref{2}] genannten Limitierungen besteht der Drang nach Möglichkeiten, Signale von höheren Frequenzen fehlerfrei Abtasten zu können. Dies zeigt sich besonders in der Hochfrequenztechnik, bei welcher die Erfassung von erzeugten Signalen eine große Rolle spielt.\\
Des Weiteren ist es möglich, die Implementierung des in [\ref{3.3}] vorgestellten ADUs zu optimieren. Es gibt bereits eine Vielzahl an effizienteren Lösungen, von denen viele durch ihre Komplexität in einem eigenständigen Echtzeitsystem jedoch nicht implementierbar sind.\\
Von ähnlicher Bedeutung ist die Realisierung eines DAUs. Hierbei liegt jedoch verstärkt Fokus auf der Genauigkeit des auszugebenden Wertes als auf der Auflösung, welche in einem Echtzeitsystem meist nicht die Dimension eines nicht-Echtzeitsystems erfordert.\\
In diesem Kontext ist auch der Aufbau des zu Grunde liegenden Systems entscheidend. Je nach System ist die Wahl des Befehlssatzes, Anzahl an Kernen, Arten des Speichers oder auch die Nutzung einer Pipeline ein Faktor, welcher die Effizienz verbessern kann. Im Kontext eines DSPs existiert hierbei bereits eine grundlegende Entscheidungsbasis, welche in [\ref{2}] erläutert und anhand von Datenblättern aufgezeigt wurde, jedoch treffen diese nicht in jedem System zu, welches sich mit Signalen befasst und muss somit je nach Anwendungsfall durchdacht werden.\\

Zusammenfassend lässt sich sagen, dass die wichtigsten Aspekte eines in Echtzeit betriebenen DSPs Speicher- sowie Energieeffizienz, Geschwindigkeit sowie Resistenz gegenüber äußeren Einflüssen sind. Bei der Entwicklung eines solchen Systems dürfen andere Aspekte nicht vernachlässigt werden, jedoch sollte auf diese Punkte verstärkt Fokus gelegt werden.