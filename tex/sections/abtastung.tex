\section{Abstatung}\label{3}
Wird ein analog vorliegendes Signal in ein digitales Signal umgewandelt, so bezeichnet sich dies als Abtastung. Dies ist ein fundamentaler und in der Regel der erste Schritt für die Verarbeitung eines Signals. Im Folgenden werden zunächst die Grundlagen und die Theorie der Abtastung erörtert. Anschließend werden die wichtigsten Eigenschaften aufgezählt und bewiesen. In Abschnitt \ref{3.3} wird auf die Bedeutung dieser Eigenschaften im Kontext eines DSPs eingegangen. Zuletzt wird die Realisierung der Abtastung am Beispiel eines DSPs beschrieben, wobei Fokus auf die Problematiken und Lösungen gelegt wird.
\subsection{Theorie}\label{3.1}
Für jegliche Verarbeitung mithilfe eines DSPs, zur Übertragung mittels beispielsweise des 'Digital Audio Broadcastings' oder zur Minimierung des Speicherverbrauchs ist die Digitalisierung ein notwendiger Schritt in der Signalverarbeitung. 

\subsubsection{Analoge/Digitale Definitions- und Wertebereiche}
\cite[p.2]{frey2008signal}


\subsection{Beweis}\label{3.2}
\subsection{Bedeutung}\label{3.3}
\subsection{Periphere Schnittstellen}\label{3.4}
\subsection{Analoger und digitaler In-/Output}\label{3.5}